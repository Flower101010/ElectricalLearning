\section{直流电路}

\subsection{基本概念}

\subsubsection{作用和组成}

电路的作用:能量的输送与转换(强电),信号的处理与传输(弱电)。

电路的基本组成:电源、负载、连接导线。

概念:电源、负载、导线、内电路、外电路、直流电路(DC)、交流电路(AC)、

\subsubsection{基本物理量}

\begin{center}
    \fbox{不随时间变化的物理量用大写字母,随时间变化的物理量用小写字母}
\end{center}

\begin{center}
    \begin{circuitikz}
        \draw (0,3) node[left]{$+$} 
            to[battery1, l=$U_s$, a=$E\uparrow$] (0,0)node[left]{$-$}
            to[short] (3,0) node[right]{$-$}
            to[lamp] (3,3) node[right]{$+$}
            to[short] (1.5,3) node[above]{$\overset{I}{\longrightarrow} $}
            to[short] (0,3);
    \end{circuitikz}
\end{center}

\begin{enumerate}
\item 电流,电流方向是正电荷的流动方向。
\[
    I = \frac{Q}{t}
\]
\[
    i = \frac{\mathrm{d} q}{\mathrm{d} t} 
\]

\item 电位 

电场力将单位正电荷从电路某一点移至参考点所消耗的电能,也就是在移动中转化非电形态
能量的电能称为该点的电位。

\item 电压

电场力将单位正电荷从电路某一点移至另一点所消耗的电能,即转化为非电形态
能量的电能称为这两点的电压。

方向:从高电位指向低电位。

\item 电动势

电源中的局外力(即非电场力)将单位正电荷从电源负极移至电源正极所换来的电能称为电
源的电动势。

方向:从电源负极指向电源正极,即低电位指向高电位。

\item 电功率

电路中单位时间所转化的电能消称为电功率,简称功率。

电源产生的功率
\[
    P_E=EI
\]
电源输出的功率
\[
    P_s=U_sI
\]
负载功率
\[
    P_L=U_LI
\]

\item 电能

在时间$t$内转化的电功率称为电能。

\[
    W=Pt
\]

\end{enumerate}

\subsubsection{电路状态}

电路状态主要有三种:通路‘开路和短路。

\subsubsection{参考方向}

\begin{center}
    \fbox{电压与电流选取的参考方向应保持一致}
\end{center}

\subsubsection{理想电路元件}

\begin{enumerate}
    \item 理性有源元件
    \begin{itemize}
        \item 电压源 电压恒定
        \item 电流源 电流恒定
    \end{itemize}
    \item 理想无源元件
    \begin{itemize}
        \item 电容
        \[
          C=\frac{q}{u}  \]
          瞬时功率
          \[
            P=ui=Cu\frac{\mathrm{d}u}{\mathrm{d}t}\]
          存储电场能
          \[
            W_e=\frac{1}{2}CU^2\]  
        \item 电感
        \[
            L=\frac{\Psi}{i}\]
          瞬时功率
          \[
            p=ui=Li\frac{\mathrm{d}i}{\mathrm{d}t}\]
            存储磁场能
            \[
                W_m=\frac{1}{2}LI^2
                \]
        \item 电阻
        \[
            R=\frac{u}{i}
            \]
        \[
            P=UI=RI^2=\frac{U^2}{R}\]
    \end{itemize}
\end{enumerate}

\subsection{基尔霍夫定律}

\subsubsection{KCL}

电路上任意结点的同一瞬间电流的代数和为零。

\begin{center}
    \begin{circuitikz}
        \draw
        (2,1.5) 
        to [short, i_=$i_1$, o-*] (2,0.5);
        \draw
        (1.14,0)
        to [short, i_=$i_2$, o-*] (2,0.5);
        \draw
        (2,0.5)
        to [short, i^=$i_3$, *-o] (2.87,0);
    \end{circuitikz}
\end{center}

\[
    i_1+i_2+i_3=0
\]

即
\[
    \sum_{k=1}^n i_k=0
\]

\subsubsection{KVL}

电路中的任意一回路,沿同一方向循行,同一瞬间电压的代数和为零。即
\[
    \sum_{k=1}^n u_k=0
\]


\subsection{支路电流法}

直接利用基尔霍夫定律,列方程组求解。

\noindent 一般步骤:
\begin{enumerate}
    \item 确定支路数,选取支路电流方向
    \item 确定结点数,列出独立的结点电流方程
    \item 确定余下所需的方程式数,列出独立的回路电压方程
    \item 解方程组
\end{enumerate}

\subsection{叠加定理}

在含有多个有源元件的线性电路中,任何一条支路上的电压或电流等于电路中各个
有源元件分别单独作用在该支路上时所产生的电压或电流的代数和。

\large{注意事项}

\normalsize
\begin{enumerate}
    \item 考虑某一有源元件单独作用时,其他有源元件$U_s=0,I_s=0$,即电压
    源代之以短路,电流源代之以开路.
    \item 注意是否与参考方向一致。
    \item 叠加定理只适用于线性电路。
    \item 叠加定理只适用于电流和电压,不适用于功率。
\end{enumerate}

\subsection{等效电源定理}

\subsubsection{戴维宁定理}

对外电路而言,任何一个线性有缘网络都可以用一个戴维宁等效电
源来替代。(等效电压源)

\begin{figure}[!ht]
    \centering
    \begin{minipage}[t]{0.4\linewidth}
        \centering
        \begin{circuitikz}
            \draw
            (0,0)
            to [V, v=$U_s$] (0,2)
            to [R=$R_1$] (2,2)
            to [I, i=$I_s$] (2,0)
            to [short] (0,0);
            \draw
            (2,2)
            to [short, -o] (4,2);
            \draw [dashed]
            (4,2)
            to [short, i^=$I_{SC}$, a_=$U_{OC}$] (4,0);
            \draw
            (4,0)
            to [short, o-] (2,0);
        \end{circuitikz}
    \caption{有源二端网络}
    \label{fig:1}
    \end{minipage}
    \begin{minipage}[t]{0.4\linewidth}
        \centering
        \begin{circuitikz}
            \draw
            (0,0)
            to [V, v=$U_s$] (0,2)
            to [R=$R_0$, -o] (4,2);

            \draw [dashed]
            (4,2)
            to [short, i^=$I_{SC}$, a_=$U_{OC}$] (4,0);
            \draw
            (4,0)
            to [short, o-] (0,0);
        \end{circuitikz}
        \caption{戴维宁等效电压电源}
        \label{fig:2}
    \end{minipage}
\end{figure}

由图易知
\[
    U_{es}=U_{oc}
\]

\[
    R_{0}=\frac{U_{eS}}{I_{SC}}=\frac{U_{oc}}{I_{sc}}  
\]

\subsubsection{诺顿定理}

对外电路而言,任何一个线性有缘网络都可以用一个诺顿等效电
源来替代。(等效电流源)

\begin{figure}[!ht]
    \centering
    \begin{minipage}[t]{0.4\linewidth}
        \centering
        \begin{circuitikz}
            \draw
            (0,0)
            to [V, v=$U_s$] (0,2)
            to [R=$R_1$] (2,2)
            to [I, i=$I_s$] (2,0)
            to [short] (0,0);
            \draw
            (2,2)
            to [short, -o] (4,2);
            \draw [dashed]
            (4,2)
            to [short, i^=$I_{SC}$, a_=$U_{OC}$] (4,0);
            \draw
            (4,0)
            to [short, o-] (2,0);
        \end{circuitikz}
    \caption{有源二端网络}
    \label{fig:3}
    \end{minipage}
    \begin{minipage}[t]{0.4\linewidth}
        \centering
        \begin{circuitikz}
            \draw
            (0,0)
            to [V, v=$U_s$] (0,2)
            to [short, -o] (4,2);

            \draw
            (2,2)
            to [R, l_=$R_0$] (2,0);

            \draw [dashed]
            (4,2)
            to [short, i^=$I_{SC}$, a_=$U_{OC}$] (4,0);
            \draw
            (4,0)
            to [short, o-] (0,0);
        \end{circuitikz}
        \caption{诺顿等效电源}
        \label{fig:4}
    \end{minipage}
\end{figure}

由图易知
\[
    I_{eS}=I_{SC}
\]

\[
    R_{0}=\frac{U_{OC}}{I_{eC}}=\frac{U_{OC}}{I_{sc}}  
\]

戴维宁等效电源与诺顿等效电源的互换(对外等效时)

\[
    I_{eS}=\frac{U_{eS}}{R_0}  
\]
