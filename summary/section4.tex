\section{变压器}

\subsection{磁路}

常用物理量:磁通量$\varPhi $,单位韦伯(Wb);
磁通量密度$B$,单位特斯拉(T);
磁场强度$H$,单位安培/米(A/m);
磁导率$\mu = \frac{B}{H}$,单位亨利/米(H/m)。

\

\noindent 磁性能:高导磁性;磁饱和性;磁滞性。

\

\noindent 磁路欧姆定律:
\begin{equation}
    \varPhi = \frac{F}{R_m}
\end{equation}

\noindent 其中$R_m$为磁路的磁阻。
\[
    R_m = R_{mc}+R_{m0}=\frac{l_c}{\mu _c A_c}+\frac{l_c}{\mu _cA_c}
\]
$F$为磁动势,等于线圈匝数$N$乘以电流$i$,即
\[F=NI\]

\subsection{电磁铁}

\subsubsection{直流电磁铁}

\textbf{电路}

\noindent 电流$I$只与线圈电压$U$和线圈电阻$R$有关,即
\[I=\frac{U}{R}\]
功耗仅有线圈电阻$R$的功耗,即
\[P=UI=I^2R=\frac{U^2}{R}\]

\textbf{吸力}

衔铁吸合后$\rightarrow R_m  \downarrow \rightarrow\varPhi  \uparrow \rightarrow$吸引力$\uparrow$

\subsubsection{交流电磁铁}

线圈中通过交变电流,由电磁感应定律知,线圈中会产感应
电动势。有以下关系:
\begin{equation}
    e = 2\pi fN\varPhi _m\sin (\omega t-90^\circ) = E_m sin (\omega t-90^\circ)
\end{equation}
在数值上,有效值$E$为:
\begin{equation}
    E = \frac{E_m}{\sqrt{2}}=\frac{2\pi Nf\varPhi _m}{\sqrt{2}} = 4.44Nf\varPhi _m 
\end{equation}
用相量表示为:
\begin{equation}
    \dot{E} = -j4.44Nf\dot{\varPhi} _m
\end{equation}

电路的功率关系与一般的交流电路相同,即
\begin{equation}
    \begin{aligned}
        S &=UI \\
        Q &=UI\sin \varphi \\
        P &=UI\cos \varphi    
    \end{aligned}
\end{equation}

$P$包含铜损$P_{Cu}$和铁损$P_{Fe}$,铁损又包含磁滞损
耗$P_h$和涡流损耗$P_e$。有以下关系:
\begin{equation}
    \begin{aligned}
        P_{Cu} &= RI^2 \\
        P &= P_{Cu}+P_{Fe} \\
        P_{Fe} &= P_h+P_e
    \end{aligned}
\end{equation}

\begin{equation}
    \begin{aligned}
        \dot{I}_{L1} &= \dot{I}_{1} - \dot{I}_{2} \\
        \dot{I}_{L2} &= \dot{I}_{2} - \dot{I}_{3} \\
        \dot{I}_{L3} &= \dot{I}_{3} - \dot{I}_{1}
    \end{aligned}
\end{equation}

\noindent 吸力

衔铁吸合后,电磁吸力的最大值和平均值基本不变,但励磁电流
变小。一般来说交流电磁铁的启动电流比工作电流大很多。

\subsection{变压器}
\begin{center}
    记$k=\frac{N_1}{N_2}$,则在不考虑损耗的情况下有下列关系
\end{center}

% Please add the following required packages to your document preamble:
% \usepackage{graphicx}
\begin{table}[!htbp]
    \centering
    \resizebox{0.5\textwidth}{!}{%
    \begin{tabular}{cc}
    \hline
    电压变换 & $\frac{U_1}{U_2}=\frac{N_1}{N_2}=k$ \\
    电流变换 & $\frac{I_1}{I_2}=\frac{N_2}{N_1}=\frac{1}{K}$ \\
    阻抗变换 & $\left\lvert Z_e\right\rvert =\frac{U_1}{I_1}=k^2\left\lvert Z_L\right\rvert $ \\
    功率传递 & $S_N=U_{N2}I_{N2}=U_{N1}I_{N1}$ \\ \hline
    \end{tabular}%
    }
\end{table}