\section{交流电路}

\subsection{正弦交流电路的基本概念}

电流瞬时表达式:

\[
    i=I_m\sin(\omega t+\psi)
\]

其中,$I_m$为电流的最大值,$\omega$为角频率,$\psi$为初相位
或相位角。\\
波形图如下

\begin{tikzpicture}
    %\draw[very thin,color=gray] (-0.1,-1.1) grid (3.9,8.1);
    \draw[->,thick] (-0.5,0) -- (8,0) node[right] {$t$};
    \draw[->,thick] (0,-1.5) -- (0,2.5) node[above] {$i$};
    
    \draw[domain=-1:7,smooth] plot(\x,{2*sin(\x r + 30)});

\end{tikzpicture}



\

周期,频率

\[
    \omega=\frac{2\pi}{T}=2\pi f,~ f=\frac{1}{T}    
\]

最大值和有效值

\[
    U = \frac{U_m}{\sqrt{2}} ,~ 
    E = \frac{E_m}{\sqrt{2}} ,~
    I=\frac{I_m}{\sqrt{2}}
\]

\

$(\omega t + \psi)$称为相位,或相位角,$\psi$称为初相位。

任意两正弦量的相位差:$\varphi = \phi_2 - \phi_1$。

\large{\textbf{相量表示法}}

\normalsize
表示正弦交流电在复平面中处于起始位置的固定矢量称为正弦交流
电的相量。
\underline{
区分最大值相量和有效值相量
}

复平面的矢量可用复数表示,矢量$\overline{OP}$的表示方法如下:
\[
    \overline{OP}=a+jb = c(\cos \psi + j\sin \psi ) = ce^{j\psi}=c\phase{\psi}
\]

为避免符号混淆,在代表交流电的符号上加上一点,以示区别。$\dot{I},\dot{U}$

\large 注意事项
\normalsize
\begin{enumerate}
    \item 相量不等于正弦交流电
    \item 只有正弦交流电才能用相量表示
    \item 只有同频率的
    正弦交流电才能进行相量运算
\end{enumerate}

\subsection{单一参数交流电路}

\begin{center}
    \textbf{电一参数交流电路的主要结论}
\end{center}

\begin{table}[!ht]
    \resizebox{\textwidth}{!}{%
    \begin{tabular}{|cc|c|c|c|}
    \hline
    \multicolumn{2}{|c|}{项目} & 电阻 & 电容 & 电感 \\ \hline
    \multicolumn{2}{|c|}{电阻或阻抗} & $R$ & $X_C=\frac{1}{2\pi fC}$ & $X_L=2\pi fL$ \\ \hline
    \multicolumn{1}{|c|}{\multirow{4}{*}{电压与电流的关系}} & 频率 & 相图 & 相同 & 相同 \\ \cline{2-5} 
    \multicolumn{1}{|c|}{} & 相位 & 相同 & $u$ 滞后$i$$90^\circ$ & $u$ 超前$i$$90^\circ$ \\ \cline{2-5} 
    \multicolumn{1}{|c|}{} & 有效值 & $U=RI$ & $U=X_CI$ & $U=X_LI$ \\ \cline{2-5} 
    \multicolumn{1}{|c|}{} & 相量式 & $\dot{U}=R\dot{I}$ & $\dot{U}=-jX_C\dot{I}$ & $\dot{U}=jX_L\dot{I}$ \\ \hline
    \multicolumn{1}{|c|}{\multirow{2}{*}{功率}} & 有功功率 & $P=UI=R^2I=\frac{U^2}{R}$ & 0 & 0 \\ \cline{2-5} 
    \multicolumn{1}{|c|}{} & 无功功率 & 0 & $Q=UI=X_CI^2=\frac{U^2}{X_C}$ & $Q=UI=X_LI^2=\frac{U^2}{X_L}$ \\ \hline
    \end{tabular}%
    }
\end{table}

\subsection{串联和并联交流电路}

\subsection{交流电路的功率和功率因数}

\subsection{电路中的谐振}
